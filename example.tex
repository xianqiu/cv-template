\documentclass{cv}

%----------------------------------------------------------------------------------------

\begin{document}

%----------------------------------------------------------------------------------------
% header
%----------------------------------------------------------------------------------------


\name{丘直}
\job{硕士研究生-21届}
\company{某个大学}
\phone{137 1234 5678}
\email{qiuzhi@mu.edu.cn}

\maketitle

\section{教育经历}

\textbf{2018.09 -- 今 \quad 硕士 \quad 某个专业 \hfill 某个大学}\\
\textbf{2014.09 -- 2018.07 \quad 本科 \quad 某个专业 \hfill 某个大学}\\
\textbf{2014.09 -- 2018.07 \quad 本科 \quad 经济学专业(辅修) \hfill 某个大学}


\section{项目经验}

\textbf{2020.1 -- 今 \quad AI辅助诊断}
\begin{itemize}
	\item 描述: 对医疗图像中的病灶进行定位、分类和分割;
	\item 使用类U-Net网络,尝试Inception/Resnet/Densenet等多种网络结构,以及多种数据增强方法,开发病灶语义分割模型,在同等假阳率下,将召回率由最初的40\%提升至88\%;
	\item 调研并尝试多种分类模型及数据增强方法,开发病灶二分类模型,将召回率由最初的75\%提升至80\%;
\end{itemize}

\textbf{2019.02 -- 2019.8 \quad 股票数据分析}
\begin{itemize}
	\item 利用XGBoost拟合港股数据建立投资模型
	\item 利用投资模型在A股中赚了1个亿
\end{itemize}

\section{获奖情况}

\textbf{2020.6 \quad 校级三好学生}\\
\textbf{2019.6 \quad 全国数学建模竞赛一等奖}

\section{主要技能}

\begin{itemize}
	\item 编程: 熟练掌握Java/Python 
	\item 写作: \LaTeX/Markdown/Word/PPT
	\item 英语: CET6
\end{itemize}


\end{document}